% LINGI2251
% Development methods
\documentclass[11pt, a4paper]{article}
\usepackage[utf8]{inputenc}
\usepackage[UKenglish]{babel}
\usepackage{graphicx}				% Use pdf, png, jpg, or eps§ with pdflatex; use eps in DVI mode
\usepackage{xcolor}
\usepackage{listings}
\usepackage{hyperref}
\usepackage{array}
\usepackage{longtable}
\usepackage{multirow}
\usepackage[babel=true]{csquotes}


\usepackage[T1]{fontenc}

\lstset{%
	basicstyle=\ttfamily\footnotesize,
	commentstyle=\color{green!90!black},
	frame=single,
	keywordstyle=\bfseries\color{blue},
	language=python,
	numberstyle=\color{gray},
%	tabsize=2,
}


\hypersetup{%
	colorlinks=true,
	linkcolor=blue,
	urlcolor=blue
}


\newcommand{\tbf}[1]{\textbf{#1}}
\newcommand{\tit}[1]{\textit{#1}}

\newcommand{\data}[1]{\textit{#1}}



\def\blurb{\textsc{Université catholique de Louvain\\
  École polytechnique de Louvain\\
  Pôle d'ingénierie informatique}}
\def\clap#1{\hbox to 0pt{\hss #1\hss}}%
\def\ligne#1{%
  \hbox to \hsize{%
    \vbox{\centering #1}}}%
\def\haut#1#2#3{%
  \hbox to \hsize{%
    \rlap{\vtop{\raggedright #1}}%
    \hss
    \clap{\vbox{\vfill\centering #2\vfill}}%
    \hss
    \llap{\vtop{\raggedleft #3}}}}%
\begin{document}

\begin{titlepage}
\thispagestyle{empty}\vbox to 1\vsize{%
  \vss
  \vbox to 1\vsize{%
    \haut{\raisebox{-2mm}{\includegraphics[width=2.5cm]{logo_epl.jpg}}}{\blurb}{\raisebox{-5mm}{\includegraphics[scale=0.20]{ingi_logo}}}
    \vfill
    \ligne{\Huge \textbf{\textsc{LINGI2251}}}
     \vspace{5mm}
    \ligne{\huge \textbf{\textsc{Development methods}}}
     \vspace{15mm}
    \ligne{\Large \textbf{\textsc{Assignment 1: The Gas Station Control System}}}
    \vspace{5mm}
    \ligne{\large{\textsc{March 23, 2015}}}
    \vfill
    \vspace{5mm}
    \ligne{%
         \textsc{Professor\\Charles Pecheur}
      }
      \vspace{10mm}
    }%
    \ligne{%
         \textsc{Michael Heraly\\Thibault Gerondal}
      }
      \vspace{5mm}
  \vss
  }
\end{titlepage}



\newpage


\section{Requirement issues}

What are the steps to create a billing account number ?


\section{Interfaces}

\subsection{Gas pump}

\begin{itemize}
\item Input :
		\begin{itemize}
		\item The trigger is pressed
		\end{itemize}

\item Output :
		\begin{itemize}
		\item \data{Gallons} [positive float] : Number of gallons purchased. 
    \item \data{Dollars} [positive float, max 999,99] : Dollar amount of the purchase.
		\end{itemize}
\end{itemize}



\subsection{Gas pump interface}

\begin{itemize}
\item Input :
		\begin{itemize}
    \item \data{Gallons} [positive float] : Number of gallons purchased (from the Gas pump).
    \item \data{Dollars} [positive float, max 999,99] : Dollar amount of the purchase (from the Gas pump).
    \item \data{Payment Token} [integer representing a {valid, invalid} payment] : Token giving information about the payment (from the Credit Card System).
    \item \data{Reset Token} [integer representing a reset operation] : When the token is received , the pump is reseted.
    \item \data{User input} [User choices] : The user can communicate his choice of payment.
    \end{itemize}

\item Output :
		\begin{itemize}
		\item \data{User output} [Strings] : Messages on the screen to guide the user through steps.
    \item \data{Gallons} [positive float] : Number of gallons purchased (to the cashier interface).
    \item \data{Dollars} [positive float, max 999,99] : Dollar amount of the purchase (to the cashier interface and to the card reader system).
		\end{itemize}
\end{itemize}


\subsection{Credit card reader}

\begin{itemize}
\item Input :
		\begin{itemize}
		\item \data{Credit card number} : credit card number read.
		\end{itemize}

\item Output :
		\begin{itemize}
		\item \data{Credit card number} : send the credit car number to the gas pump interface.
		\item \data{Invalid token} : send an invalid token to the gas pump interface if the credit card number cannot be read correctly.
		\end{itemize}
\end{itemize}



\subsection{Credit card system}

\begin{itemize}
\item Input :
		\begin{itemize}
		\item \data{credit card number} : receive the credit card number from the gas pump interface.
		\item \data{purchase amount} : receive the purchase amount from the gas pump interface.
		\end{itemize}
\end{itemize}



\subsection{Cashier}

\begin{itemize}
\item Input :
		\begin{itemize}
		\item \data{cash} [positive float] : receive cash payments from customers.
		\end{itemize}

\item Output :
		\begin{itemize}
		\item \data{change} [positive float] : give money as change for the payment.
		\item \data{payment complete} : indicate that the payment is complete to the cashier's interface.
		\end{itemize}
\end{itemize}



\subsection{Cashier's interface}

\begin{itemize}
\item Input :
		\begin{itemize}
		\item \data{type of payment} : the cashier indicates whether : 
					\begin{itemize}
					\item the payment has to be added to a monthly bill.
					\item the payment is done by credit card.
					\end{itemize}
		\item \data{billing account number} : the cashier enters the billing account number of the customer if the payment is to be made by monthly bill.
		\item \data{credit card account number} : the cashier enters the credit card account number if the customer pays by credit card.
		\item \data{payment complete} : the cashier indicates that the payment is complete.
		\end{itemize}

\item Output :
		\begin{itemize}
		\item \data{display purchase price} : display the purchase price (if the payment is to be made by cash, or by monthly bill).
		\item \data{display error message} : display an error message if the billing account number entered is invalid.
    \item \data{Reset Token} [integer representing a reset operation] : When the token is received , the pump is reseted.
		\end{itemize}
\end{itemize}



\section{State of the system}

The GSCS stores the following data :
\begin{itemize}
\item Related to the gas station :
		\begin{itemize}
		\item \data{gallons purchased}
		\item \data{purchase amount}
		\end{itemize}

\item Related to the customers :
		\begin{itemize}
		\item \data{Personal informations} : name, address, phone number, billing account number.
		\item \data{Purchase informations} : how many gallons, the purchase amount, and the date.
		\item \data{Purchase status} : does the customer have a due amount, type of payment, date of the purchase, date of the payment.
		\end{itemize}
\end{itemize}


\section{Data-flow diagram}

\section{Class diagram}

\section{Sequence diagrams}

\section{State diagrams}

\begin{figure}[h]
\centering
\includegraphics[width=\textwidth]{StateDiagram_Cashier_interface.pdf}
\caption{State diagram -- Cashier's interface}
\end{figure}

\end{document}
