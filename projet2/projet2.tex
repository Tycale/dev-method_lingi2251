% LINGI2251
% Development methods
\documentclass[11pt, a4paper]{article}
\usepackage[utf8]{inputenc}
\usepackage[UKenglish]{babel}
\usepackage{graphicx}				% Use pdf, png, jpg, or eps§ with pdflatex; use eps in DVI mode
\usepackage{xcolor}
\usepackage{listings}
\usepackage{hyperref}
\usepackage{array}
\usepackage{longtable}
\usepackage{multirow}
\usepackage[babel=true]{csquotes}


\usepackage[T1]{fontenc}

\lstset{%
	basicstyle=\ttfamily\footnotesize,
	commentstyle=\color{green!90!black},
	frame=single,
	keywordstyle=\bfseries\color{blue},
	language=python,
	numberstyle=\color{gray},
%	tabsize=2,
}


\hypersetup{%
	colorlinks=true,
	linkcolor=blue,
	urlcolor=blue
}


\newcommand{\tbf}[1]{\textbf{#1}}
\newcommand{\tit}[1]{\textit{#1}}

\newcommand{\data}[1]{\textit{#1}}
\newcommand{\state}[1]{\textsf{#1}}



\def\blurb{\textsc{Université catholique de Louvain\\
  École polytechnique de Louvain\\
  Pôle d'ingénierie informatique}}
\def\clap#1{\hbox to 0pt{\hss #1\hss}}%
\def\ligne#1{%
  \hbox to \hsize{%
    \vbox{\centering #1}}}%
\def\haut#1#2#3{%
  \hbox to \hsize{%
    \rlap{\vtop{\raggedright #1}}%
    \hss
    \clap{\vbox{\vfill\centering #2\vfill}}%
    \hss
    \llap{\vtop{\raggedleft #3}}}}%
\begin{document}

\begin{titlepage}
\thispagestyle{empty}\vbox to 1\vsize{%
  \vss
  \vbox to 1\vsize{%
    \haut{\raisebox{-2mm}{\includegraphics[width=2.5cm]{logo_epl.jpg}}}{\blurb}{\raisebox{-5mm}{\includegraphics[scale=0.20]{ingi_logo}}}
    \vfill
    \ligne{\Huge \textbf{\textsc{LINGI2251}}}
     \vspace{5mm}
    \ligne{\huge \textbf{\textsc{Development methods}}}
     \vspace{15mm}
    \ligne{\Large \textbf{\textsc{Assignment 2: The Dinoco GSCS}}}
    \vspace{5mm}
    \ligne{\large{\textsc{April 20, 2015}}}
    \vfill
    \vspace{5mm}
    \ligne{%
         \textsc{Professor\\Charles Pecheur}
      }
      \vspace{10mm}
    }%
    \ligne{%
         \textsc{Michael Heraly\\Thibault Gerondal}
      }
      \vspace{5mm}
  \vss
  }
\end{titlepage}



\newpage


\section{Architectural Design}
\subsection{Hierarchical decomposition}

\begin{center}
\centerline{\includegraphics[width=1.3\textwidth]{hierarchical.png}}
\end{center}

\subsection{Roles and interactions}

Our system is composed of two parts : ``Accountancy'' and ``Gas Pump''. The first one is related to accounting. And the second one is about the logistic of the gas pump. This hierarchy allows to easily extend the activities of \textbf{Dinoco GSCS}.\\




In each of these two sections, we defined three main components : ``Interfaces'', ``Controllers'', ``Storages''.\\


Here are the descriptions of the system components :
\begin{description}
\item[CashierInterface] is responsible for displaying the graphical user interface to the cashier.
\item[AccountController] is responsible for the access to accounts informations (relation to AccountancyStorage).
\item[CashierController] is responsible for the interactions between the CashierInterface and the AccountController (e.g. open a monthly bill, pay a monthly bill, etc.). It also permits the cashier to check the fuel stock from the PumpController (which will retrieve the information from the FuelStorage). 
\item[DevicesController] is responsible for the interactions between the CreditCardReader and the CreditCardSystem. 
\item[AccountancyStorage] is responsible for storing the accounts informations.
\item[CreditCardReader] is responsible for reading credit card number (wrapping the existing API for the device).
\item[CreditCardSystem] is responsible for requesting payment (wrapping the existing system API).
\item[PumpInterface] is responsible for displaying informations to the customer.
\item[PumpController] is responsible for the interactions between the PumpInterface and the Dinoco GSCS.
\item[FuelStorage] is responsible for storing the fuel informations.
\end{description}

\section{Detailed design}
\subsection*{Class diagram}

\subsection*{short description}

\subsubsection*{CashierController}
The CashierController is the main controller. It is the entry point of the application. It will initialize the PumpController, the AccountController, the PurchaseController and finally its interface (CashierInterface).

\subsubsection*{CashierInterface}
This class is responsible of drawing the graphical user interface for the cashier.

\subsubsection*{PurchaseFactory}
The PurchaseController is responsible of creating and saving new Purchase. If the purchase is a CreditPurchase, the controller will interact with the CreditCardController for directly settle the payment.

\subsubsection*{Purchase}
This class represents a purchase. This class is abstract.

\subsubsection*{CreditPurchase}
Specialized class for representing a purchase made via credit card.

\subsubsection*{AccountPurchase}
Specialized class for representing a purchase that is not directly paid. The purchase will be associated to a monthly bill account (Account class).

\subsubsection*{CashPurchase}
Specialized class for representing a purchase made via cash.

\subsubsection*{Account}
This class is responsible for keeping up to date the accountancy of a customer.

\subsubsection*{AccountController}
This controller is responsible for creating new Account and getting existing Account.

\subsubsection*{Customer}
Class representing a customer.

\subsubsection*{AccountancyStorage}
This class is responsible of the persistence (to keep in a database) of classes : Account, Purchase and Customer.

\subsubsection*{CreditCardController}
This class will interact with the CreditCardReader and the CreditCardSystem. It will offer a simple way to to request a payment by manipulating the external APIs of the credit card company. We need to know the pumpNumber to enable the right card reader associated to a gaz pump.

\subsubsection*{CreditCardSystem}
This class will wrap the external API of the credit card system.

\subsubsection*{CreditCardReader}
This class will wrap the external API of the credit card reader.

\subsubsection*{PumpInterface}
This class is responsible for displaying messages and listen to choice made by the user.

\subsubsection*{PumpController}
This controller manages everything related to the Fuel stock. Since the user can pay directly by credit card at a pump, it can call the PurchaseFactory to handle the payment.

\subsubsection*{Fuel}
Class for representing fuel stock.

\subsubsection*{FuelStorage}
This class is responsible of the persistence (to keep in a database) of the Fuel stock.

\end{document}
